\documentclass[12pt,letterpaper]{article}
\usepackage[utf8]{inputenc}
\usepackage[spanish]{babel}
\usepackage{graphicx}
\usepackage[left=2cm,right=2cm,top=2cm,bottom=2cm]{geometry}
\usepackage{graphicx} % figuras
% \usepackage{subfigure} % subfiguras
\usepackage{float} % para usar [H]
\usepackage{amsmath}
%\usepackage{txfonts}
\usepackage{stackrel} 
\usepackage{graphicx}
\usepackage{subfig}
\usepackage{hyperref}
\usepackage{multirow}
\usepackage{enumerate} % enumerados
\renewcommand{\labelitemi}{$-$}
\renewcommand{\labelitemii}{$\cdot$}
% \author{}
% \title{Caratula}
\begin{document}

% Fancy Header and Footer
% \usepackage{fancyhdr}
% \pagestyle{fancy}
% \cfoot{}
% \rfoot{\thepage}
%

% \usepackage[hidelinks]{hyperref} % CREA HYPERVINCULOS EN INDICE
  
% \author{}
\title{Caratula}

\begin{titlepage}
    \begin{center}
    \begin{figure}[htb]
    \begin{center}
    \includegraphics[width=3.5cm]{./img/upt.jpg}
    \end{center}
    \end{figure}
    
    \vspace*{0.15in}
    \begin{Large}
    \textbf{UNIVERSIDAD PRIVADA DE TACNA}\\
    \end{Large}
    
    \vspace*{0.1in}
    \begin{Large}
    \textbf{FACULTAD DE INGENIERIA} \\
    \end{Large}
    
    \vspace*{0.1in}
    \begin{Large}
    \textbf{ESCUELA PROFESIONAL DE INGENIERIA DE SISTEMAS} \\
    \end{Large}
    
    \vspace*{0.5in}
    \begin{Large}
    \textbf{TITULO:}\\
    \end{Large}
    

\vspace*{0.1in}
\begin{Large}
    PRACTICA DE LABORATORIO 04: ELABORACION DE DASHBOARDS EN POWER BI\\
\end{Large}

\vspace*{0.3in}
\begin{Large}
\textbf{Curso:} \\
\end{Large}

\vspace*{0.1in}
\begin{large}
    Inteligencia De Negocios\\
\end{large}

\vspace*{0.3in}
\begin{Large}
\textbf{Docente:} \\
\end{Large}

\vspace*{0.1in}
\begin{large}
Ing. Patrick Cuadros Quiroga\\
\end{large}

\vspace*{0.2in}
\vspace*{0.1in}
\begin{large}
\textbf{Alumno:} \\
\begin{flushleft}
 Herrera Amezquita, Derian Francisco	\hfill	(2017059489) \\


\end{flushleft}
\end{large}
\vspace*{0.1in}
\begin{large}
Tacna - Perú\\
\end{large}
\vspace*{0.1in}
\begin{large}
2021\\
\end{large}

\end{center}

\end{titlepage}



\tableofcontents % INDICE
\thispagestyle{empty} % INDICE SIN NUMERO
\newpage
\setcounter{page}{1} % REINICIAR CONTADOR DE PAGINAS DESPUES DEL INDICE


\section{Objetivos}
A
\section{Requerimientos}
Conocimientos
\\Para el desarrollo de esta práctica se requerirá de los siguientes conocimientos básicos:
\\- Conocimientos básicos de administración de base de datos Microsoft SQL Server.
\\- Conocimientos básicos de SQL.
\\\\Software
\\\\Asimismo se necesita los siguientes aplicativos:
\\\\- Microsoft SQL Server 2017 o superior
\\\\- Base de datos AdventureWorksDW 2017 o superior
\\\\- Power BI Desktop.
\\\\- Tener una cuenta Microsoft registrada en el Portal de Power Bi 

\section{Consideraciones Iniciales}
Microsoft Power BI es un conjunto de aplicaciones para el análisis empresarial, que permite unificar diferentes
fuentes de datos, configura y analiza datos que son presentados de manera sencilla en tablas e informes, que
pueden ser consultados de una manera muy fácil y atractiva en tiempo real por usuarios e integrantes de una
misma empresa u organización.

\section{Desarrollo}
Paso 1: Para esta guía utilizaremos el cubo creado en la guía anterior. Inicie Power BI Desktop, busque y
seleccione la opción Get Data
\begin{center}
    \includegraphics[width=10cm]{img/power.png}  
\end{center}
\begin{center}
    \includegraphics[width=14cm]{img/1.png}  
\end{center}
Paso 2: Dentro de los resources seleccionaremos SQL Server database
\begin{center}
    \includegraphics[width=7cm]{img/2.png}  
\end{center}
Paso 3: Utilice el nombre de host o localhost para conectarse
\begin{center}
    \includegraphics[width=12cm]{img/3.png}  
\end{center}
Vamos a seleccionar Adventure Works DW2017
\begin{center}
    \includegraphics[width=15cm]{img/4.png}
    \vspace{2cm}  
\end{center}
\begin{center}
    \includegraphics[width=10cm]{img/5.png}  
\end{center}
Paso 4: Una vez conectado tendremos en nuestro lado dos toolbox, uno denominado VISUALIZATONS y
otro denominado FIELDS.
\\\\\\En FIELDS debe mostrar la Fact Table de Internet Sales y las dimensiones asociadas según las guías
previas de cubos. 
\begin{center}
    \includegraphics[width=8cm]{img/paso4.png}  
    \vspace{2cm}
\end{center}
Paso 5: Vamos a crear nuestro primer reporte. Seleccionaremos una gráfica de barras, en segundo lugar
Sales Amount, Calendar Year y English Product Name. (Debe hacerlo en ese orden).
\\\\La gráfica resultante es la siguiente:
\begin{center}
    \includegraphics[width=17cm]{img/6.png}
    \vspace{2cm}  
\end{center}
Paso 7: Elimine la gráfica anterior y procederá a seleccionar gráfica de barras, en segundo lugar Sales
Amount, English Product Name y Calendar Year. (Debe hacerlo en ese orden).
\begin{center}
    \includegraphics[width=15cm]{img/7.png}  
\end{center}
La gráfica cambiará, lo que indica que el orden de agregado es importante para las visualizaciones, aún
habiendo seleccionado los mismos datos.
\\\\Paso 8: Cree un nuevo reporte.
\\\\Podemos crear un dashboard con gráficos simultáneos. Arrastre dos gráficas y seleccione una de ella
para establecer las propiedades.
\begin{center}
    \includegraphics[width=15cm]{img/8.png}  
\end{center}
\begin{center}
    \includegraphics[width=11cm]{img/graph.png}  
\end{center}
\begin{center}
    \includegraphics[width=18cm]{img/9.png}  
\end{center}
\begin{center}
    \includegraphics[width=18cm]{img/10.png}  
\end{center}
\begin{center}
    \includegraphics[width=18cm]{img/11.png}  
\end{center}
Paso 10: Seleccione una de los valores de la gráfica de la izquierda para ver el comportamiento:
\begin{center}
    \includegraphics[width=18cm]{img/12.png}  
\end{center}
Paso 11: Ahora crearemos un mapa que muestre la proporción de ventas por zona geográfica. Arrastre
un Mapa y una tabla
\begin{center}
    \includegraphics[width=13cm]{img/pic.png}  
\end{center}
\begin{center}
    \includegraphics[width=11cm]{img/13.png}  
\end{center}
Paso 12: Ahora generaremos una gráfica de área.
\begin{center}
    \includegraphics[width=13cm]{img/14.png}  
\end{center}
En primer lugar seleccione una nueva página y agregue una gráfica de área.
\\\\Luego seleccione las medidas que se van a mostrar en el gráfico:
\begin{center}
    \includegraphics[width=8cm]{img/pic1.png}  
\end{center}
Ordene el resultado por año de manera ascendente.
\begin{center}
    \includegraphics[width=15cm]{img/15.png}  
\end{center}
Paso 13: Agregaremos una gráfica de líneas. Vamos a seleccionar desde la tabla de hecho a Sales
Amount.
\begin{center}
    \includegraphics[width=15cm]{img/17.png}  
\end{center}
A continuación agregaremos Calendar Year desde Order Date y luego English Country Region Name.
Debe realizarse en este orden o el resultado será diferente.
\begin{center}
    \includegraphics[width=15cm]{img/16.png}  
\end{center}
Paso 14: Puede definir los nombres de las hojas para indicar el tipo de reporte y la información.
Establezca nombres descriptivos según la información que usted quiere facilitar.
\begin{center}
    \includegraphics[width=14cm]{img/18.png}  
\end{center}
Paso 15: Crearemos un reporte (gráfico de barras) con filtrado básico. Seleccionar Sales Amount, English
Country Region Name y Order date/Calendar Year. Buscará la sección Basic Filtering y marcará Canada /
United Kingdom.
\begin{center}
    \includegraphics[width=17cm]{img/19.png}  
\end{center}
Paso 16: La siguiente gráfica es una Stacked Column Chart. Los atributos que utilizaremos son Sales
Amount vs English ProductName vs Order Date/Calendar Year
\begin{center}
    \includegraphics[width=17cm]{img/20.png}
    \vspace{2cm}  
\end{center}
Ahora incluya un filtro. Buscaremos productos que hayan vendido arriba de los \$400,000
\begin{center}
    \includegraphics[width=15cm]{img/21.png}  
\end{center}
Paso 17: Incluya un Table con los siguientes campos:
\\\\Sales Amount, English Product Name y Calendar Year
\\\\Seleccione Show Data
\begin{center}
    \includegraphics[width=15cm]{img/22.png}  
\end{center}
Podrá visualizar el detalle de ventas
\begin{center}
    \includegraphics[width=17cm]{img/23.png}  
\end{center}
Paso 18: Cambie la orientación del reporte:
\begin{center}
    \includegraphics[width=15cm]{img/24.png}  
\end{center}
Paso 19: Exporte su reporte para visualización
\begin{center}
    \includegraphics[width=11cm]{img/25.png}  
\end{center}



\section{Análisis de resultados}
Utilizando la base de datos Chinook, investigue cómo generar la dimensión de tiempo y luego, crear los
siguientes reportes (tome en cuenta que se necesita conocer los valores en dinero):
\\\\Se requiere saber cuáles son los artistas que más han vendido en la plataforma
\\\\• Ventas por país contra año
\\• Ventas por país (mapa)
\\• Ventas realizadas por artista y año.
\\• Genere 4 reportes adicionales utilizando los datos que usted considere relevantes.
\\• Investigue cómo visualizar los reportes que ha creado en dispositivos móviles.








\end{document}

